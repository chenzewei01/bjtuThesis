\chapter{使用方法}
限于篇幅,这边只挑一部分讲,期望你自己去探索!
\section{论文信息录入}
请在data/info.tex文件录入论文信息,更详细地,请见info.tex文件里的注释。
\section{目录生成}
目录由\verb|\tableofcontents|命令自动生成,需要两次xelatex编译(所以第一次编译后没见到目录,不要紧,再编译一次即可)。
\section{正文撰写}
请阅读文献 《\href{http://mirrors.ustc.edu.cn/CTAN/info/lshort/english/lshort.pdf}{The Not So Short Introduction to ~\LaTeXe{}~}》\cite{oetiker1995not},中译版有点老《 \href{http://mirrors.hustunique.com/CTAN/info/lshort/chinese/lshort-zh-cn.pdf}{一份不太简短的~\LaTeXe{}~ 介绍}》
\section{参考文献管理}
这部分重点介绍吧!\par
通过以下几个步骤,可以轻松完成参考文献的生成。\par
\begin{enumerate}
  \item 在谷歌学术搜索中检索到文献后,在文献条目区域单击导入BibTeX 选项,页面中出现文献的引用信息;
  \item 将文献引用信息的内容复制之后,添加到data文件夹下的ref.bib中;
  \item 当标注参考文献时,在需要标注的地方输入\verb|\cite{}| 指令,花括号内输入参考文献引用信息中的第一行信息即可(常常为文献的缩略信息);
  \item 用BibTex编译main.aux文件(该文件由xelatex编译main.tex产生,所以若没有main.aux,请编译main.tex),之后再用xelatex编译两次main.tex,保证参考文献条目的插入及目录的完整;
\end{enumerate}
\par
这里介绍的只是一个小技巧,BibTex具有强大的参考文献管理功能,有兴趣的同学可以进一步研究。
\section{增加宏包}
为了方便管理,你可以在data/package.tex增加你所需要的宏包。
